\documentclass{acm_proc_article-sp}

% UTF8 encoding and scalable fonts
\usepackage[utf8]{inputenc}
\usepackage[T1]{fontenc}

\usepackage{microtype}
\usepackage{graphicx}
\usepackage{subfigure}
\usepackage{booktabs}
\usepackage{listings}
\usepackage[hyphens]{url}
\usepackage[hidelinks]{hyperref}

% \usepackage{doi}
% \setlength{\paperheight}{11.69in}

\begin{document}

% Leave as is
\conferenceinfo{5th Seminar on Research Trends in Media Informatics (RTMI '13).}{\\February 2013, Ulm University, Ulm, Germany.}
\CopyrightYear{2013}

\title{Interaction Issues during Autonomous Driving}

\numberofauthors{1}
\author{
\alignauthor
Tamino P.S.M. Hartmann\\
       \affaddr{Institute of Media Informatics}\\
       \affaddr{Ulm University}\\
       \affaddr{Ulm, Germany}\\
       \email{tamino.hartmann@uni-ulm.de}
}

\maketitle
\begin{abstract}
This paper represents a short overview over control hand-off issues for automated automotive applications – meaning self-driving cars.
We will shortly take a look at existing knowledge and then extrapolate problems and solutions that might be of interest.
% Might want to reword that :P
Where required, we will fill knowledge gaps with careful presumptions.
\end{abstract}

\keywords{autonomous vehicle, self-driving car, driver-vehicle interaction, control hand-off}

\section{Introduction}
% I think the introduction might be too broad – might want to put it more to the point and maybe put the stuff now here in a short <history> section or something.

Autonomous vehicles have a surprisingly long history, although they have only recently become technologically feasible.
First widely proposed by Norman Bel Geddes in his book Magic Motorways \cite{geddes2009magic} and at the World Fair in 1939, driverless cars were first thought to be cars that were controlled by technology within the road.
Only in the 1980's did the steering control move from the road into the car itself.
There it has stayed for now, although due to the significant adoption of ubiquitous networking, the first systems have now been proposed where individual cars communicate with each other to further improve congestion and transport flow – effectively moving control back to a common intelligence.

While the first estimates concerning the adoption of autonomous vehicles proved to be widely optimistic, we have now come into a time where the technology is capable of fulfilling this science fiction dream.
Google's self-driving car \cite{www:google_car} has been widely reported on, although by far not the first project within the field.
First large steps were made by the introduction of the DARPA-funded Autonomous Land Vehicle project in the United States of America.

% TODO: Correct ref.
Indeed the first cars with partial autonomous capabilities have entered commercial production and are publicly available, as can be seen in section \ref{current_tech}.
These capabilities include systems that keep a car traveling within its lane, park the car, and systems that react to emergency situations, for example by beginning breaking the vehicle even before the driver has had a chance to react.

As these systems continue to increase in modern cars, the role of the driver is transforming from the single controlling entity to a more copilot-like role.
However the current state of affairs hints that some interaction will always be required by the driver, for example when the autonomous drive system is confronted with a situation it cannot handle.
This is where a potential massive problem exists: the hand-off of control from the vehicle to the driver.

We will therefore in the following take a closer look at the current state of the technology, how working prototypes handle the issue, and what might be future solutions to arising problems.
Our reasoning will be based on own contributions and existing work done in automation, such as airplane and train automation, where we will study whether findings can be transferred to autonomous cars.

\section{Current Technological State}

In this section we will take a look at existing work on the topic of interaction issues between autonomous vehicles and their operators.
We will begin by shortly locking at alternative fields where automation is already actively in use, mainly aircraft and trains.
Then we will give a brief topic-specific general overview of existing technology and solutions in the automotive field.

\subsection{Existing Work}
% Continue here

Little work exists at the time of writing this paper.
We believe this to be because of two reasons: the relatively young age of the technology and the commercial secrecy of vehicle manufactures.

An article in the Huffington Post \cite{www:huffington_post} brought the work of Clifford Nass to our attention who has begun working on interaction issues.
However no published work has yet been publicly been made available.

A draft also exists that parallels the proposed paper, but to our knowledge has not been finished yet \cite{cummings:authority}.
It will also be incorporated into the proposed paper, especially as it has a similar approach to the topic.

This section will provide an overview of any existing and relevant work we find.
Due to the nature of the investing companies, that means that we will mainly have to focus on published newspaper articles for a glimpse of existing work.

\subsection{Current Technology}
\label{current_tech}
% TODO: Change structure as proposed by Weber

Although most companies that have begun developing autonomous car technology remain relatively secretive, some information is publicly available.
Therefore we will take a look at some systems of interest that are known to exist at this time, including the few systems already commercially available.
We will briefly highlight their technological standing and what can be deduced for interaction issues from how they have implemented the technology.

\subsubsection{Autonomous Trains and Airplanes}
% Note that trains haven't been considered in the related work yet, might want to concentrate more on them instead of airplanes.

An important point to remember when considering autonomous systems independent of the vehicle they control however is the capabilities of their human counterparts.
Pilots and train engineers are for the most part highly trained.
Chauffeurs for cars come near to the concept, but autonomous cars will in general be used by the general population.
A population that has a varied range of experience, physical handicaps, and for whom driving is a tool to use instead of a profession.

Here we will take a look if lessons can be learned from systems already in use for trains and airplanes.
This includes issues that have already been identified and have caused accidents.
Specifically we will study if some lessons can be learned and if their solutions can be applied to self-driving cars.

\subsubsection{Daimler / Mercedes-Benz}

Daimler is advancing into the field of self-driving cars with its Intelligent Drive technology \cite{www:daimler_intelligent_drive}.

Also interesting is the Attention Assistance function of a S-class Mercedes \cite{www:mercedes_attention_assist}.
It can be significant for interaction issues as it extends the monitoring the car does from the outside world to the occupants of the vehicle.
We will propose using such a function to improve how the hand-off issue can be handled.

\subsubsection{Google}

Google is the company that the public mostly associates with self-driving technology.
Only a few articles exist and no official publication, but there are some videos that might be of concern to the topic.
One of these shows another aspect of interacting with an autonomous car: the video where a legally blind person uses an autonomous car to drive \cite{www:google_blind}.
This should be considered for the paper as the capability of such a driver to interact with the car is severely handicapped and therefore should be considered.

\section{Current and Future Issues}

This section explains the issues that already exist and issues that might and or will arise due to autonomous vehicles to the best of our abilities.

\subsection{General Assumptions}
%reword

Here we will briefly highlight what assumptions we make concerning autonomous vehicles.
This includes for example that the system will never actively cause an accident (due to fail-safe behavior).
We will also define the role we assume the driver to have and how he will fulfill it.
This should be done to allow us to focus only on the interaction aspects and not on the general technological hurdles that need to be overcome.

\subsection{Driver Readiness and Capabilities}
% psychological issues

A major issue is the attention span of drivers.
It has been shown that humans tend to lose focus when supervising an autonomous system that seldom requires interaction.
Therefore measuring the driver alertness and adapting how and if the car hands control over to the driver based on it should be considered also.

An increase in automation also implies that drivers will miss out in experience with road situations that are nowadays still common.
Mainly this means that drivers will become increasingly less capable of reacting correctly in the event that the autonomous system requires their assistance, even if they have been notified to the best of the systems capabilities beforehand.

This will also lead us to look at how to best convey required situational awareness to a driver in a possibly time-critical situation.
That again ties in with using attention assistance systems to monitor what information the driver requires to react best.

This section will look at the human component of the interaction with the autonomous vehicle and what issues come from it.
We will look at the problems that arise from an adoption of autonomous technology from a non-technical standpoint.

\subsection{Control Hand-off Issues}
\label{hand_off_issue}
% This is actually the main topic, move up and display more prominently

Even when the hand-off is successful (implying that the driver is in full control of the vehicle and aware of the fact), he will take a moment to understand the situation – moments he might not have.
It is even possible that due to the high automation, drivers become unfamiliar with driving, lowering their capabilities to react in a timely and correct fashion to an emergency.
This could actually increase the risk of an accident happening.

\begin{quote}
The most dangerous moment in a self-driving car involves no immediate or obvious peril.
It is not when, say, the computer must avoid a vehicle swerving into its lane or navigate some other recognizable hazard of the road -- a patch of ice, or a clueless pedestrian stepping into traffic.
It is when something much more routine takes place: The computer hands over control of the vehicle to a human being.
In that instant, the human must quickly rouse herself from whatever else she might have been doing while the computer handled the car and focus her attention on the road.
As scientists now studying this moment have come to realize, the hand-off is laden with risks.
\end{quote}
From \cite{www:huffington_post}.

% Not sure if the quotes can be used thus, but keep for now...
\begin{quote}
This brings up the most challenging obstacle on our road to the autonomous- driving future: managing the handoff.
For as long as anyone, even Google, is willing to predict, cars will by necessity be semiautonomous; human drivers will still have to play some role.
But figuring out what that role will be is complicated.
Are we pilots or copilots? How far out of the loop can we be taken?
\end{quote}
From \cite{www:wired}.

Here we will take a look at the most important interaction issue – the hand-off of control from the vehicle to the driver.
The two quotes are an excellent starting point for any of our work.
This will include a comparison of how it is done in current technology and how it will evolve as self-driving cars become better and driver assistance is required less frequently.
We will also consider limiting factors such as when the driver cannot be trusted to be fully capable of controlling the vehicle and whether the system should evaluate if it has a better chance of avoiding problems by not initiating a hand-off in the first place.

\section{Solutions}
% Todo: come up with some novel solutions that might actually be feasibel – also consider changing the way drivers are trained! Also add in psychological things to consider, such as the switch found in trains to check if the driver is still alert

In this section we will discuss proposed solutions to the aforementioned issues, if available.
We will also offer up our own solutions based on existing work and technologies.
This section will most likely be highly speculative and thus will have no existing work to be based on.

\section{Possible Future Work}
% Might want to point to the work being done by the other papers and work that should be done; alternatively, issues that MUST be solved for the technology to be safe

Here we will offer a future outlook and where further work needs to be done.
This includes systems where control of the vehicle is per default not with the driver – implying autonomous vehicles beyond their adoptive period where control is still with the driver.

\section{Conclusion}

This section will be a summary of our findings.

% do not change the bibliography style
\bibliographystyle{abbrv}
\bibliography{sources}  

\balancecolumns

\end{document}
